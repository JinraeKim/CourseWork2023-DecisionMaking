\documentclass{article}

\usepackage{amsmath, amsthm, amssymb, amsfonts}
\usepackage{thmtools}
\usepackage{graphicx}
\usepackage{setspace}
\usepackage{geometry}
\usepackage{float}
\usepackage{hyperref}
\usepackage[utf8]{inputenc}
\usepackage[english]{babel}
\usepackage{framed}
\usepackage[dvipsnames]{xcolor}
\usepackage{tcolorbox}

\colorlet{LightGray}{White!90!Periwinkle}
\colorlet{LightOrange}{Orange!15}
\colorlet{LightGreen}{Green!15}

\newcommand{\HRule}[1]{\rule{\linewidth}{#1}}

\declaretheoremstyle[name=Assumption,]{thmsty}
\declaretheorem[style=thmsty,numberwithin=section]{assumption}
\tcolorboxenvironment{assumption}{colback=LightOrange}

\declaretheoremstyle[name=Theorem,]{thmsty}
\declaretheorem[style=thmsty,numberwithin=section]{theorem}
\tcolorboxenvironment{theorem}{colback=LightGray}

\declaretheoremstyle[name=Proposition,]{prosty}
\declaretheorem[style=prosty,numberlike=theorem]{proposition}
\tcolorboxenvironment{proposition}{colback=LightOrange}

\declaretheoremstyle[name=Principle,]{prcpsty}
\declaretheorem[style=prcpsty,numberlike=theorem]{principle}
\tcolorboxenvironment{principle}{colback=LightGreen}

\setstretch{1.2}
\geometry{
    textheight=9in,
    textwidth=5.5in,
    top=1in,
    headheight=12pt,
    headsep=25pt,
    footskip=30pt
}

% ------------------------------------------------------------------------------

\begin{document}

% ------------------------------------------------------------------------------
% Cover Page and ToC
% ------------------------------------------------------------------------------

\title{ \normalsize \textsc{}
		\\ [2.0cm]
		\HRule{1.5pt} \\
		\LARGE \textbf{\uppercase{2023 SNU Course Work - Decision Making}
		\HRule{2.0pt} \\ [0.6cm] \LARGE{HW \#2} \vspace*{10\baselineskip}}
		}
\date{}
\author{\textbf{Authors} \\ 
  Jinrae Kim with SNU FDCL members\\
		}

\maketitle
\newpage

% ------------------------------------------------------------------------------

\section{Problem 2}
\begin{align}
  \label{eq:x1_dot}
  \dot{x}_{1} &= a x_{1} - x_{1}^{3} + x_{2},
  \\
  \dot{x}_{2} &= u.
\end{align}
$x_{1_{d}}:[0, \infty) \to \mathbb{R}$, desired trajectory of $x_{1}$, is given.
$a \in \mathbb{R}$ is assumed to be a constant scalar.

\begin{align}
  z_{1}(t) &:= x_{1}(t) - x_{1_{d}}(t)
  \\
  z_{2}(t) &:= x_{2}(t) - x_{2_{d}}(t)
\end{align}
Here, $x_{2_{d}}(t)$ is the desired $x_{2}$ by regarding $x_{2}$ in \eqref{eq:x1_dot} as a virtual input,
which will be designed in each section.

\subsection{(a) Backstepping control (a=1 is known)}
$x_{1_{d}}$ is assumed to be twice differentiable.

Lyapunov function for the outer-loop control:
\begin{equation}
  V_{1} := \frac{1}{2} z_{1}^{2}.
\end{equation}
\begin{align}
  \dot{V}_{1} &:= z_{1} \dot{z}_{1}
  \\
              &= z_{1} (x_{2} + a x_{1} - x_{1}^{3} - \dot{x}_{1_{d}}).
\end{align}
Let $x_{2_{d}} := -k_{1} z_{1} -(a x_{1}-x_{1}^{3} - \dot{x}_{1_{d}})$ with a positive constant $k_{1} \in \mathbb{R}_{>0}$.
Then,
\begin{equation}
  \dot{V}_{1} = -k_{1} z_{1}^{2} + z_{1} z_{2},
\end{equation}
and if $z_{2} = 0$ (exact tracking of $x_{2}$ to $x_{2_{d}}$), the virtual input $x_{2_{d}}$ makes $z_{1} = 0$ a globally asymptotically stable equilibrium point of the outer-loop control system.

Note that the time derivative of $x_{2_{d}}$ is given as follows,
\begin{equation}
  \dot{x}_{2_{d}} = -k_{1} (\dot{x}_{1} - \dot{x}_{1_{d}}) - (a \dot{x}_{1} - 3x_{1}^{2} \dot{x}_{1} - \ddot{x}_{1_{d}}),
\end{equation}
and this quantity requires the knowledge of $a$.

Now, define the Lyapunov function for the whole control system:
\begin{equation}
  V_{2} := V_{1} + \frac{1}{2} z_{2}^{2}.
\end{equation}
\begin{align}
  \dot{V}_{2} &:= -k_{1} z_{1}^{2} + z_{1} z_{2} + z_{2} \dot{z}_{2}
  \\
              &= -k_{1} z_{1}^{2} + z_{1} z_{2} + z_{2} (u - \dot{x}_{2_{d}})
\end{align}
Letting $u = \dot{x}_{2_{d}} - z_{1} - k_{2} z_{2}$ implies
\begin{equation}
  \dot{V}_{2} = -k_{1} z_{1}^{2} - k_{2} z_{2}^{2} \prec 0,
\end{equation}
which makes $(z_{1}, z_{2}) = (0, 0)$ a globally asymptotically stable equilibrium point,
and thus $x_{1}(t) \to x_{1_{d}} (t)$ as $t \to \infty$.

\subsection{(b) Adaptive backstepping control (a=1 is unknown)}
\begin{assumption}
  $x_{1_{d}}$ is bounded.
\end{assumption}

We will introduce a new state $\hat{a}(t) \in \mathbb{R}$ for adaptive control.
Since $a$ is unknown, we modify the virtual input as
$\hat{x}_{2_{d}} = x_{2_{d}} \vert_{a = \hat{a}} = -k_{1}z_{1} - (\hat{a} x_{1} - x_{1}^{3} - \dot{x}_{1_{d}}) $
and corresponding error $\hat{z}_{2} := x_{2} - \hat{x}_{2_{d}}$.
Define the estimation error $\tilde{a} := \hat{a} - a$.
Also, note for $\hat{\dot{x}}_{1} := \dot{x}_{1} \vert_{a = \hat{a}}$ that
$\hat{\dot{x}}_{1} - \dot{x}_{1} = \tilde{a} x_{1} $
Use the following Lyapunov function candidate:
\begin{equation}
  V := \frac{1}{2} z_{1}^{2} + \frac{1}{2} \hat{z}_{2}^{2} + \frac{1}{2} \gamma^{-1} \tilde{a}^{2}.
\end{equation}
\begin{equation}
  \dot{V} = z_{1} \dot{z}_{1} + \hat{z}_{2} \dot{\hat{z}}_{2} + \gamma^{-1} \tilde{a} \dot{\hat{a}}
\end{equation}
\begin{align}
  \label{eq:z_1}
  z_{1} \dot{z}_{1} &= z_{1} (\hat{\dot{x}}_{1} - (\hat{\dot{x}}_{1} - \dot{x}_{1}) - \dot{x}_{1_{d}})
  \\
                    &= z_{1} ((\hat{a} x_{1} - x_{1}^{3} + x_{2}) - \tilde{a} x_{1} - \dot{x}_{1_{d}})
                    \\
                    &= z_{1} ((\hat{a} x_{1} - x_{1}^{3} + \hat{x}_{2_{d}}) - \tilde{a} x_{1} - \dot{x}_{1_{d}}) + z_{1} \hat{z}_{2}
                    \\
                    &= -k_{1} z_{1}^{2} + z_{1} \hat{z}_{2} - z_{1} \tilde{a} x_{1}.
\end{align}

Note that
\begin{align}
  \dot{\hat{x}}_{2_{d}} &= -k_{1} (\dot{x}_{1} - \dot{x}_{1_{d}}) - (\dot{\hat{a}} x_{1} + \hat{a} \dot{x}_{1} - 3x_{1}^{2} \dot{x}_{1} - \ddot{x}_{1_{d}})
  \\
                        &= -k_{1} (\hat{\dot{x}}_{1} - \dot{x}_{1_{d}}) - (\dot{\hat{a}} x_{1} + \hat{a} \hat{\dot{x}}_{1} - 3x_{1}^{2} \hat{\dot{x}}_{1} - \ddot{x}_{1_{d}})
                        + (k_{1} + \hat{a} - 3x_{1}^{2}) (\hat{\dot{x}}_{1} - \dot{x}_{1})
                        \\
                        &= -k_{1} (\hat{\dot{x}}_{1} - \dot{x}_{1_{d}}) - (\dot{\hat{a}} x_{1} + \hat{a} \hat{\dot{x}}_{1} - 3x_{1}^{2} \hat{\dot{x}}_{1} - \ddot{x}_{1_{d}})
                        + (k_{1} + \hat{a} - 3x_{1}^{2}) \tilde{a} x_{1},
\end{align}
and define its estimate as
\begin{equation}
  \hat{\dot{\hat{x}}}_{2_{d}} := \dot{\hat{x}}_{2_{d}} \vert_{a = \hat{a}} = -k_{1} (\hat{\dot{x}}_{1} - \dot{x}_{1_{d}}) - (\dot{\hat{a}} x_{1} + \hat{a} \hat{\dot{x}}_{1} - 3x_{1}^{2} \hat{\dot{x}}_{1} - \ddot{x}_{1_{d}}),
\end{equation}
and the estimation error is $\hat{\dot{\hat{x}}}_{2_{d}} - \dot{\hat{x}}_{2_{d}} = -(k_{1}+\hat{a}-3x_{1}^{2})\tilde{a} x_{1} $.

Then,
\begin{align}
  \hat{z}_{2} \dot{\hat{z}}_{2} &= \hat{z}_{2} (u - \dot{\hat{x}}_{2_{d}})
  \\
                                &= \hat{z}_{2} (u - \hat{\dot{\hat{x}}}_{2_{d}} - (k_{1} + \hat{a} - 3x_{1}^{2})\tilde{a} x_{1}).
\end{align}
Letting $u = \hat{\dot{\hat{x}}}_{2_{d}} - z_{1} - k_{2}\hat{z}_{2}$ implies
\begin{align}
  \label{eq:z_2_hat}
  \hat{z}_{2} \dot{\hat{z}}_{2} = -z_{1}\hat{z}_{2} -k_{2} \hat{z}_{2}^{2} - \hat{z}_{2} (k_{1} + \hat{a} - 3 x_{1}^{2}) \tilde{a} x_{1}.
\end{align}

Therefore,
the following adaptive law,
\begin{equation}
  \dot{\hat{a}} = \gamma (z_{1} + \hat{z}_{2}(k_{1} + \hat{a} - 3x_{1}^{2})),
\end{equation}
implies
\begin{align}
  \dot{V} = -k_{1}z_{1}^{2} - k_{2} \hat{z}_{2}^{2} - (z_{1} + \hat{z}_{2}(k_{1} + \hat{a} - 3x_{1}^{2}) ) \tilde{a} x_{1} + \gamma^{-1} \tilde{a} \dot{\hat{a}}
  = -k_{1}z_{1}^{2} - k_{2} \hat{z}_{2}^{2} \leq 0.
\end{align}

Note that $V$ is non-increasing,
and therefore, $z_{1}$, $\hat{z}_{2}$, and $\tilde{a}$ is bounded.
This also implies that $\hat{a}$ is bounded.
By assumption, $x_{1} = z_{1} + x_{1_{d}}$ is also bounded.
Now, observe this:
\begin{equation}
  \ddot{V} = -2(k_{1} z_{1}\dot{z}_{1} + k_{2} \hat{z}_{2} \dot{\hat{z}}_{2}).
\end{equation}
From \eqref{eq:z_1} and \eqref{eq:z_2_hat},
$z_{1} \dot{z}_{1}$ and $\hat{z}_{2} \dot{\hat{z}}_{2}$ contains $x_{1}$, $z_{1}$, $\hat{z}_{2}$, $\hat{a}$, $\tilde{a}$ only,
and therefore, $\ddot{V}$ is bounded.

By Barbalat's lemma, this implies that $\dot{V} \to 0$ as $t \to \infty$, which shows that $(z_{1}, \hat{z}_{2}) = (0, 0)$ is a globally asymptotically stable equilibrium point.

\begin{theorem}[Barbalat's lemma, modified]
  Given $f: [0, \infty) \to \mathbb{R}$,
  if $f(t)$ has a finite limit and $\ddot{f}$ is bounded,
  then $\dot{f} \to 0$ as $t \to \infty$.
\end{theorem}

% \begin{proposition}
%     This is a proposition.
% \end{proposition}
%
% \begin{principle}
%     This is a principle.
% \end{principle}

% Maybe I need to add one more part: Examples.
% Set style and colour later.

% \subsection{Pictures}
%
% \subsection{Citation}

\newpage

% ------------------------------------------------------------------------------
% Reference and Cited Works
% ------------------------------------------------------------------------------

% \bibliographystyle{IEEEtran}
% \bibliography{References.bib}

% ------------------------------------------------------------------------------

\end{document}
